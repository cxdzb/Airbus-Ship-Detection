一、背景绪论

二、数据处理

1、数据预览

船舶图像:(19万张左右)

\includegraphics[width=2.22708in,height=2.22708in]{media/image1.png}

数据目标:(保存在csv表格中)

\includegraphics[width=4.14514in,height=3.23542in]{media/image2.png}

这是所得数据的rle编码,解码效果如下:

\includegraphics[width=2.22083in,height=2.22569in]{media/image3.png}

\begin{enumerate}
\def\labelenumi{\arabic{enumi}.}
\setcounter{enumi}{1}
\item
  数据解码

  Rle编码(Run Length
  Encoding),它的原理是通过检测统计数据流中重复的位或字符序列,并用它们出现的次数和每次出现的个数形成新的代码。从而达到数据压缩的目的。

  例如,``5 3 10
  4''表示,目标所在位置包括:第5个像素点开始连续的3个像素点和第10个像素点开始连续的4个像素点。

  Rle解码算法:

  新建一张全0图像,对其修改对应的目标像素为1。

  Rle解码代码如下:
\end{enumerate}

\begin{longtable}[]{@{}l@{}}
\toprule
\begin{minipage}[t]{0.97\columnwidth}\raggedright\strut
\textbf{def} rle\_to\_array(img, rles):

l, w = img.shape{[}0{]}, img.shape{[}1{]}\\
x = np.asarray(img).reshape((l, w, 3))\\
y = np.zeros(l * w, dtype=np.uint8)\\[2\baselineskip]\textbf{for} rle
\textbf{in} rles.values:\\
\textbf{if} rle \textbf{is} np.nan:\\
\textbf{break\\
}rle = rle.split(\textbf{' '})\\
starts, lengths = {[}np.asarray(x, dtype=int) \textbf{for} x \textbf{in}
(rle{[}0:{]}{[}::2{]}, rle{[}1:{]}{[}::2{]}){]}\\
starts -= 1\\
ends = starts + lengths\\
\textbf{for} s, e \textbf{in} zip(starts, ends):\\
y{[}s:e{]} = 1\\[2\baselineskip]y = y.reshape((l, w)).T.reshape((l, w,
1))\\
\textbf{return} x, y\strut
\end{minipage}\tabularnewline
\bottomrule
\end{longtable}

\begin{enumerate}
\def\labelenumi{\arabic{enumi}.}
\setcounter{enumi}{1}
\item
  数据清洗

  删除数据中的残缺图像、模糊图像、重复图像。

  \includegraphics[width=1.59375in,height=1.58819in]{media/image4.png}\includegraphics[width=1.60764in,height=1.60208in]{media/image5.png}\includegraphics[width=2.50000in,height=2.26389in]{media/image6.png}

  对粘连的小船舶数据,实行部分舍弃或合并。

  \includegraphics[width=2.11111in,height=2.11667in]{media/image7.png}
\item
  数据平衡

  由于图像中的船只数量并不固定,因此我们将其分类:
\end{enumerate}

\begin{enumerate}
\def\labelenumi{\arabic{enumi}.}
\item
  、先按照有无船只分类

  \includegraphics[width=3.34167in,height=2.10486in]{media/image8.png}

  无船图像大概有15万张,有船图像大概有4万张。

  \includegraphics[width=1.83056in,height=1.05556in]{media/image9.png}

  \includegraphics[width=1.88125in,height=1.39722in]{media/image10.png}\includegraphics[width=1.84375in,height=1.37153in]{media/image11.png}
\item
  、再根据船只数量分类

  \includegraphics[width=3.50486in,height=2.44653in]{media/image12.png}

  特点为:图像中船只数量越多,图像数量越少。
\item
  、将不同船只数量的图像按比例结合

  将每种图像都取出一部分放入预准备数据中,使各种数据比例大致相同,得到一个平衡的预准备数据。(无船图像数据中包含内容较少,不取)
\end{enumerate}

\begin{enumerate}
\def\labelenumi{\arabic{enumi}.}
\setcounter{enumi}{1}
\item
  数据生成

  由于原始数据是RGB图像数据,所占内存较大,为了防止内存溢出,制作一个数据生成器。

  定义一个循环,不断地用随机数选择图像数据,选择后将目标数据进行解码处理得到L图像数据,最后将RGB图像和L图像抛出。

  代码如下:
\end{enumerate}

\begin{longtable}[]{@{}l@{}}
\toprule
\vtop{\hbox{\strut \textbf{def} generator(imgs, results, batch\_size,
seed=\textbf{None}):}\hbox{\strut \textbf{if}
seed:}\hbox{\strut np.random.seed(seed)}\hbox{\strut ImageId,
EncodedPixels = results{[}\textbf{"ImageId"}{]},
results{[}\textbf{"EncodedPixels"}{]}}\hbox{\strut \textbf{while
True}:}\hbox{\strut samples = np.random.choice(imgs,
size=batch\_size)}\hbox{\strut X, Y = {[}{]},
{[}{]}}\hbox{\strut \textbf{for} s \textbf{in} samples:}\hbox{\strut img
= Image.open(img\_path)}\hbox{\strut rles = EncodedPixels{[}ImageId ==
s{]}}\hbox{\strut x, y = rle\_to\_array(img,
rles)}\hbox{\strut X.append(x)}\hbox{\strut Y.append(y)}\hbox{\strut X,
Y = np.asarray(X) / 255, np.asarray(Y)}\hbox{\strut \textbf{yield} (X,
Y)}}\tabularnewline
\bottomrule
\end{longtable}

\begin{enumerate}
\def\labelenumi{\arabic{enumi}.}
\setcounter{enumi}{1}
\item
  数据增强

  使用keras库的ImageDataGenerator,设置多种参数使图像进行各种变换而产生更多的数据。

  常用参数设置:

  Rotation\_range:图像随即旋转的角度范围。

  width\_shift/height\_shift:图像在水平或垂直方向上的平移范围。

  Shear\_range:随机错切变换的范围。

  Zoom\_range:图像随机缩放的范围。

  Horizontal\_flip:随机将一半图像水平翻转。

  Fill\_mode:用于填充新创建像素的方法。

  例:

  \includegraphics[width=2.48333in,height=1.15833in]{media/image13.png}

  \includegraphics[width=2.47431in,height=1.15069in]{media/image14.png}
\end{enumerate}

\begin{enumerate}
\setcounter{enumi}{2}
\item
  模型建立
\end{enumerate}

\begin{enumerate}
\def\labelenumi{\arabic{enumi}.}
\item
  U-net简介

  \includegraphics[width=5.76944in,height=3.74931in]{media/image15.png}

  模型的结构十分的对称,左边为一系列连续地卷积池化,而右边则相反,为一系列连续地上采样卷积,而图中灰色箭头则是残差连接,将浅层特征与深层特征结合,避免了特征的损失,此模型因结构形似``U''形,而被称为U-net。
\item
  模型搭建
\end{enumerate}

\begin{enumerate}
\def\labelenumi{\arabic{enumi}.}
\item
  、卷积池化
\end{enumerate}

\begin{longtable}[]{@{}l@{}}
\toprule
\vtop{\hbox{\strut c = Conv2D(8, (3, 3), activation=\textbf{'relu'},
padding=\textbf{'same'})(x)}\hbox{\strut c = Conv2D(8, (3, 3),
activation=\textbf{'relu'}, padding=\textbf{'same'})(c)}\hbox{\strut p =
MaxPooling2D((2, 2))(c)}}\tabularnewline
\bottomrule
\end{longtable}

\begin{enumerate}
\def\labelenumi{\arabic{enumi}.}
\item
  、拼接+反卷积
\end{enumerate}

\begin{longtable}[]{@{}l@{}}
\toprule
\vtop{\hbox{\strut c2 = Conv2DTranspose(8, (2, 2), strides=(2, 2),
padding=\textbf{'same'})(c)}\hbox{\strut m = concatenate({[}c1, c2{]},
axis=3)}\hbox{\strut c = Conv2D(8, (3, 3), activation=\textbf{'relu'},
padding=\textbf{'same'})(m)}\hbox{\strut c = Conv2D(8, (3, 3),
activation=\textbf{'relu'}, padding=\textbf{'same'})(c)}}\tabularnewline
\bottomrule
\end{longtable}

Conv2DTranspose可用UpSampling2D+Conv2D代替。

\begin{enumerate}
\def\labelenumi{\arabic{enumi}.}
\item
  、添加批标准化层
\end{enumerate}

\begin{longtable}[]{@{}l@{}}
\toprule
c = BatchNormalization()(c)\tabularnewline
\bottomrule
\end{longtable}

统一分布,加强网络的泛化能力;

使用更高的学习率,加快训练速度;

充当Dropout,防止过拟合。

\begin{enumerate}
\def\labelenumi{\arabic{enumi}.}
\item
  、优化器与损失
\end{enumerate}

optimizer使用Adam优化器,loss取(1-IoU),IoU是预测图像与真实图像的重合率,IoU越大,loss越小。

\includegraphics[width=4.45694in,height=2.62500in]{media/image16.png}

\begin{enumerate}
\def\labelenumi{\arabic{enumi}.}
\item
  、输入输出
\end{enumerate}

\begin{longtable}[]{@{}l@{}}
\toprule
\begin{minipage}[t]{0.97\columnwidth}\raggedright\strut
input = Input((512, 512, 3))

output = Conv2D(1, (1, 1), activation=\textbf{'sigmoid'})(c)\strut
\end{minipage}\tabularnewline
\bottomrule
\end{longtable}

输入RGB图像,输出01图像,实质是像素点矩阵。

\begin{enumerate}
\def\labelenumi{\arabic{enumi}.}
\item
  、模型结果
\end{enumerate}

\begin{longtable}[]{@{}l@{}}
\toprule
\begin{minipage}[t]{0.97\columnwidth}\raggedright\strut
\_\_\_\_\_\_\_\_\_\_\_\_\_\_\_\_\_\_\_\_\_\_\_\_\_\_\_\_\_\_\_\_\_\_\_\_\_\_\_\_\_\_\_\_\_\_\_\_\_\_\_\_\_\_\_\_\_\_\_\_\_\_\_\_\_\_\_\_\_\_\_\_\_\_\_\_\_\_\_\_\_\_\_\_\_\_\_\_\_\_\_\_\_\_\_\_\_\_

Layer (type) Output Shape Param \# Connected to

==================================================================================================

input\_1 (InputLayer) (None, 512, 512, 3) 0

\_\_\_\_\_\_\_\_\_\_\_\_\_\_\_\_\_\_\_\_\_\_\_\_\_\_\_\_\_\_\_\_\_\_\_\_\_\_\_\_\_\_\_\_\_\_\_\_\_\_\_\_\_\_\_\_\_\_\_\_\_\_\_\_\_\_\_\_\_\_\_\_\_\_\_\_\_\_\_\_\_\_\_\_\_\_\_\_\_\_\_\_\_\_\_\_\_\_

conv2d\_1 (Conv2D) (None, 512, 512, 8) 224 input\_1{[}0{]}{[}0{]}

\_\_\_\_\_\_\_\_\_\_\_\_\_\_\_\_\_\_\_\_\_\_\_\_\_\_\_\_\_\_\_\_\_\_\_\_\_\_\_\_\_\_\_\_\_\_\_\_\_\_\_\_\_\_\_\_\_\_\_\_\_\_\_\_\_\_\_\_\_\_\_\_\_\_\_\_\_\_\_\_\_\_\_\_\_\_\_\_\_\_\_\_\_\_\_\_\_\_

batch\_normalization\_1 (BatchNor (None, 512, 512, 8) 32
conv2d\_1{[}0{]}{[}0{]}

\_\_\_\_\_\_\_\_\_\_\_\_\_\_\_\_\_\_\_\_\_\_\_\_\_\_\_\_\_\_\_\_\_\_\_\_\_\_\_\_\_\_\_\_\_\_\_\_\_\_\_\_\_\_\_\_\_\_\_\_\_\_\_\_\_\_\_\_\_\_\_\_\_\_\_\_\_\_\_\_\_\_\_\_\_\_\_\_\_\_\_\_\_\_\_\_\_\_

conv2d\_2 (Conv2D) (None, 512, 512, 8) 584
batch\_normalization\_1{[}0{]}{[}0{]}

\_\_\_\_\_\_\_\_\_\_\_\_\_\_\_\_\_\_\_\_\_\_\_\_\_\_\_\_\_\_\_\_\_\_\_\_\_\_\_\_\_\_\_\_\_\_\_\_\_\_\_\_\_\_\_\_\_\_\_\_\_\_\_\_\_\_\_\_\_\_\_\_\_\_\_\_\_\_\_\_\_\_\_\_\_\_\_\_\_\_\_\_\_\_\_\_\_\_

batch\_normalization\_2 (BatchNor (None, 512, 512, 8) 32
conv2d\_2{[}0{]}{[}0{]}

\_\_\_\_\_\_\_\_\_\_\_\_\_\_\_\_\_\_\_\_\_\_\_\_\_\_\_\_\_\_\_\_\_\_\_\_\_\_\_\_\_\_\_\_\_\_\_\_\_\_\_\_\_\_\_\_\_\_\_\_\_\_\_\_\_\_\_\_\_\_\_\_\_\_\_\_\_\_\_\_\_\_\_\_\_\_\_\_\_\_\_\_\_\_\_\_\_\_

max\_pooling2d\_1 (MaxPooling2D) (None, 256, 256, 8) 0
batch\_normalization\_2{[}0{]}{[}0{]}

\_\_\_\_\_\_\_\_\_\_\_\_\_\_\_\_\_\_\_\_\_\_\_\_\_\_\_\_\_\_\_\_\_\_\_\_\_\_\_\_\_\_\_\_\_\_\_\_\_\_\_\_\_\_\_\_\_\_\_\_\_\_\_\_\_\_\_\_\_\_\_\_\_\_\_\_\_\_\_\_\_\_\_\_\_\_\_\_\_\_\_\_\_\_\_\_\_\_

conv2d\_3 (Conv2D) (None, 256, 256, 16) 1168
max\_pooling2d\_1{[}0{]}{[}0{]}

\_\_\_\_\_\_\_\_\_\_\_\_\_\_\_\_\_\_\_\_\_\_\_\_\_\_\_\_\_\_\_\_\_\_\_\_\_\_\_\_\_\_\_\_\_\_\_\_\_\_\_\_\_\_\_\_\_\_\_\_\_\_\_\_\_\_\_\_\_\_\_\_\_\_\_\_\_\_\_\_\_\_\_\_\_\_\_\_\_\_\_\_\_\_\_\_\_\_

batch\_normalization\_3 (BatchNor (None, 256, 256, 16) 64
conv2d\_3{[}0{]}{[}0{]}

\_\_\_\_\_\_\_\_\_\_\_\_\_\_\_\_\_\_\_\_\_\_\_\_\_\_\_\_\_\_\_\_\_\_\_\_\_\_\_\_\_\_\_\_\_\_\_\_\_\_\_\_\_\_\_\_\_\_\_\_\_\_\_\_\_\_\_\_\_\_\_\_\_\_\_\_\_\_\_\_\_\_\_\_\_\_\_\_\_\_\_\_\_\_\_\_\_\_

conv2d\_4 (Conv2D) (None, 256, 256, 16) 2320
batch\_normalization\_3{[}0{]}{[}0{]}

\_\_\_\_\_\_\_\_\_\_\_\_\_\_\_\_\_\_\_\_\_\_\_\_\_\_\_\_\_\_\_\_\_\_\_\_\_\_\_\_\_\_\_\_\_\_\_\_\_\_\_\_\_\_\_\_\_\_\_\_\_\_\_\_\_\_\_\_\_\_\_\_\_\_\_\_\_\_\_\_\_\_\_\_\_\_\_\_\_\_\_\_\_\_\_\_\_\_

batch\_normalization\_4 (BatchNor (None, 256, 256, 16) 64
conv2d\_4{[}0{]}{[}0{]}

\_\_\_\_\_\_\_\_\_\_\_\_\_\_\_\_\_\_\_\_\_\_\_\_\_\_\_\_\_\_\_\_\_\_\_\_\_\_\_\_\_\_\_\_\_\_\_\_\_\_\_\_\_\_\_\_\_\_\_\_\_\_\_\_\_\_\_\_\_\_\_\_\_\_\_\_\_\_\_\_\_\_\_\_\_\_\_\_\_\_\_\_\_\_\_\_\_\_

max\_pooling2d\_2 (MaxPooling2D) (None, 128, 128, 16) 0
batch\_normalization\_4{[}0{]}{[}0{]}

\_\_\_\_\_\_\_\_\_\_\_\_\_\_\_\_\_\_\_\_\_\_\_\_\_\_\_\_\_\_\_\_\_\_\_\_\_\_\_\_\_\_\_\_\_\_\_\_\_\_\_\_\_\_\_\_\_\_\_\_\_\_\_\_\_\_\_\_\_\_\_\_\_\_\_\_\_\_\_\_\_\_\_\_\_\_\_\_\_\_\_\_\_\_\_\_\_\_

conv2d\_5 (Conv2D) (None, 128, 128, 32) 4640
max\_pooling2d\_2{[}0{]}{[}0{]}

\_\_\_\_\_\_\_\_\_\_\_\_\_\_\_\_\_\_\_\_\_\_\_\_\_\_\_\_\_\_\_\_\_\_\_\_\_\_\_\_\_\_\_\_\_\_\_\_\_\_\_\_\_\_\_\_\_\_\_\_\_\_\_\_\_\_\_\_\_\_\_\_\_\_\_\_\_\_\_\_\_\_\_\_\_\_\_\_\_\_\_\_\_\_\_\_\_\_

batch\_normalization\_5 (BatchNor (None, 128, 128, 32) 128
conv2d\_5{[}0{]}{[}0{]}

\_\_\_\_\_\_\_\_\_\_\_\_\_\_\_\_\_\_\_\_\_\_\_\_\_\_\_\_\_\_\_\_\_\_\_\_\_\_\_\_\_\_\_\_\_\_\_\_\_\_\_\_\_\_\_\_\_\_\_\_\_\_\_\_\_\_\_\_\_\_\_\_\_\_\_\_\_\_\_\_\_\_\_\_\_\_\_\_\_\_\_\_\_\_\_\_\_\_

conv2d\_6 (Conv2D) (None, 128, 128, 32) 9248
batch\_normalization\_5{[}0{]}{[}0{]}

\_\_\_\_\_\_\_\_\_\_\_\_\_\_\_\_\_\_\_\_\_\_\_\_\_\_\_\_\_\_\_\_\_\_\_\_\_\_\_\_\_\_\_\_\_\_\_\_\_\_\_\_\_\_\_\_\_\_\_\_\_\_\_\_\_\_\_\_\_\_\_\_\_\_\_\_\_\_\_\_\_\_\_\_\_\_\_\_\_\_\_\_\_\_\_\_\_\_

batch\_normalization\_6 (BatchNor (None, 128, 128, 32) 128
conv2d\_6{[}0{]}{[}0{]}

\_\_\_\_\_\_\_\_\_\_\_\_\_\_\_\_\_\_\_\_\_\_\_\_\_\_\_\_\_\_\_\_\_\_\_\_\_\_\_\_\_\_\_\_\_\_\_\_\_\_\_\_\_\_\_\_\_\_\_\_\_\_\_\_\_\_\_\_\_\_\_\_\_\_\_\_\_\_\_\_\_\_\_\_\_\_\_\_\_\_\_\_\_\_\_\_\_\_

max\_pooling2d\_3 (MaxPooling2D) (None, 64, 64, 32) 0
batch\_normalization\_6{[}0{]}{[}0{]}

\_\_\_\_\_\_\_\_\_\_\_\_\_\_\_\_\_\_\_\_\_\_\_\_\_\_\_\_\_\_\_\_\_\_\_\_\_\_\_\_\_\_\_\_\_\_\_\_\_\_\_\_\_\_\_\_\_\_\_\_\_\_\_\_\_\_\_\_\_\_\_\_\_\_\_\_\_\_\_\_\_\_\_\_\_\_\_\_\_\_\_\_\_\_\_\_\_\_

conv2d\_7 (Conv2D) (None, 64, 64, 64) 18496
max\_pooling2d\_3{[}0{]}{[}0{]}

\_\_\_\_\_\_\_\_\_\_\_\_\_\_\_\_\_\_\_\_\_\_\_\_\_\_\_\_\_\_\_\_\_\_\_\_\_\_\_\_\_\_\_\_\_\_\_\_\_\_\_\_\_\_\_\_\_\_\_\_\_\_\_\_\_\_\_\_\_\_\_\_\_\_\_\_\_\_\_\_\_\_\_\_\_\_\_\_\_\_\_\_\_\_\_\_\_\_

batch\_normalization\_7 (BatchNor (None, 64, 64, 64) 256
conv2d\_7{[}0{]}{[}0{]}

\_\_\_\_\_\_\_\_\_\_\_\_\_\_\_\_\_\_\_\_\_\_\_\_\_\_\_\_\_\_\_\_\_\_\_\_\_\_\_\_\_\_\_\_\_\_\_\_\_\_\_\_\_\_\_\_\_\_\_\_\_\_\_\_\_\_\_\_\_\_\_\_\_\_\_\_\_\_\_\_\_\_\_\_\_\_\_\_\_\_\_\_\_\_\_\_\_\_

conv2d\_8 (Conv2D) (None, 64, 64, 64) 36928
batch\_normalization\_7{[}0{]}{[}0{]}

\_\_\_\_\_\_\_\_\_\_\_\_\_\_\_\_\_\_\_\_\_\_\_\_\_\_\_\_\_\_\_\_\_\_\_\_\_\_\_\_\_\_\_\_\_\_\_\_\_\_\_\_\_\_\_\_\_\_\_\_\_\_\_\_\_\_\_\_\_\_\_\_\_\_\_\_\_\_\_\_\_\_\_\_\_\_\_\_\_\_\_\_\_\_\_\_\_\_

batch\_normalization\_8 (BatchNor (None, 64, 64, 64) 256
conv2d\_8{[}0{]}{[}0{]}

\_\_\_\_\_\_\_\_\_\_\_\_\_\_\_\_\_\_\_\_\_\_\_\_\_\_\_\_\_\_\_\_\_\_\_\_\_\_\_\_\_\_\_\_\_\_\_\_\_\_\_\_\_\_\_\_\_\_\_\_\_\_\_\_\_\_\_\_\_\_\_\_\_\_\_\_\_\_\_\_\_\_\_\_\_\_\_\_\_\_\_\_\_\_\_\_\_\_

max\_pooling2d\_4 (MaxPooling2D) (None, 32, 32, 64) 0
batch\_normalization\_8{[}0{]}{[}0{]}

\_\_\_\_\_\_\_\_\_\_\_\_\_\_\_\_\_\_\_\_\_\_\_\_\_\_\_\_\_\_\_\_\_\_\_\_\_\_\_\_\_\_\_\_\_\_\_\_\_\_\_\_\_\_\_\_\_\_\_\_\_\_\_\_\_\_\_\_\_\_\_\_\_\_\_\_\_\_\_\_\_\_\_\_\_\_\_\_\_\_\_\_\_\_\_\_\_\_

conv2d\_9 (Conv2D) (None, 32, 32, 128) 73856
max\_pooling2d\_4{[}0{]}{[}0{]}

\_\_\_\_\_\_\_\_\_\_\_\_\_\_\_\_\_\_\_\_\_\_\_\_\_\_\_\_\_\_\_\_\_\_\_\_\_\_\_\_\_\_\_\_\_\_\_\_\_\_\_\_\_\_\_\_\_\_\_\_\_\_\_\_\_\_\_\_\_\_\_\_\_\_\_\_\_\_\_\_\_\_\_\_\_\_\_\_\_\_\_\_\_\_\_\_\_\_

batch\_normalization\_9 (BatchNor (None, 32, 32, 128) 512
conv2d\_9{[}0{]}{[}0{]}

\_\_\_\_\_\_\_\_\_\_\_\_\_\_\_\_\_\_\_\_\_\_\_\_\_\_\_\_\_\_\_\_\_\_\_\_\_\_\_\_\_\_\_\_\_\_\_\_\_\_\_\_\_\_\_\_\_\_\_\_\_\_\_\_\_\_\_\_\_\_\_\_\_\_\_\_\_\_\_\_\_\_\_\_\_\_\_\_\_\_\_\_\_\_\_\_\_\_

conv2d\_10 (Conv2D) (None, 32, 32, 128) 147584
batch\_normalization\_9{[}0{]}{[}0{]}

\_\_\_\_\_\_\_\_\_\_\_\_\_\_\_\_\_\_\_\_\_\_\_\_\_\_\_\_\_\_\_\_\_\_\_\_\_\_\_\_\_\_\_\_\_\_\_\_\_\_\_\_\_\_\_\_\_\_\_\_\_\_\_\_\_\_\_\_\_\_\_\_\_\_\_\_\_\_\_\_\_\_\_\_\_\_\_\_\_\_\_\_\_\_\_\_\_\_

batch\_normalization\_10 (BatchNo (None, 32, 32, 128) 512
conv2d\_10{[}0{]}{[}0{]}

\_\_\_\_\_\_\_\_\_\_\_\_\_\_\_\_\_\_\_\_\_\_\_\_\_\_\_\_\_\_\_\_\_\_\_\_\_\_\_\_\_\_\_\_\_\_\_\_\_\_\_\_\_\_\_\_\_\_\_\_\_\_\_\_\_\_\_\_\_\_\_\_\_\_\_\_\_\_\_\_\_\_\_\_\_\_\_\_\_\_\_\_\_\_\_\_\_\_

conv2d\_transpose\_1 (Conv2DTrans (None, 64, 64, 64) 32832
batch\_normalization\_10{[}0{]}{[}0{]}

\_\_\_\_\_\_\_\_\_\_\_\_\_\_\_\_\_\_\_\_\_\_\_\_\_\_\_\_\_\_\_\_\_\_\_\_\_\_\_\_\_\_\_\_\_\_\_\_\_\_\_\_\_\_\_\_\_\_\_\_\_\_\_\_\_\_\_\_\_\_\_\_\_\_\_\_\_\_\_\_\_\_\_\_\_\_\_\_\_\_\_\_\_\_\_\_\_\_

concatenate\_1 (Concatenate) (None, 64, 64, 128) 0
batch\_normalization\_8{[}0{]}{[}0{]}

conv2d\_transpose\_1{[}0{]}{[}0{]}

\_\_\_\_\_\_\_\_\_\_\_\_\_\_\_\_\_\_\_\_\_\_\_\_\_\_\_\_\_\_\_\_\_\_\_\_\_\_\_\_\_\_\_\_\_\_\_\_\_\_\_\_\_\_\_\_\_\_\_\_\_\_\_\_\_\_\_\_\_\_\_\_\_\_\_\_\_\_\_\_\_\_\_\_\_\_\_\_\_\_\_\_\_\_\_\_\_\_

conv2d\_11 (Conv2D) (None, 64, 64, 64) 73792
concatenate\_1{[}0{]}{[}0{]}

\_\_\_\_\_\_\_\_\_\_\_\_\_\_\_\_\_\_\_\_\_\_\_\_\_\_\_\_\_\_\_\_\_\_\_\_\_\_\_\_\_\_\_\_\_\_\_\_\_\_\_\_\_\_\_\_\_\_\_\_\_\_\_\_\_\_\_\_\_\_\_\_\_\_\_\_\_\_\_\_\_\_\_\_\_\_\_\_\_\_\_\_\_\_\_\_\_\_

conv2d\_12 (Conv2D) (None, 64, 64, 64) 36928 conv2d\_11{[}0{]}{[}0{]}

\_\_\_\_\_\_\_\_\_\_\_\_\_\_\_\_\_\_\_\_\_\_\_\_\_\_\_\_\_\_\_\_\_\_\_\_\_\_\_\_\_\_\_\_\_\_\_\_\_\_\_\_\_\_\_\_\_\_\_\_\_\_\_\_\_\_\_\_\_\_\_\_\_\_\_\_\_\_\_\_\_\_\_\_\_\_\_\_\_\_\_\_\_\_\_\_\_\_

conv2d\_transpose\_2 (Conv2DTrans (None, 128, 128, 32) 8224
conv2d\_12{[}0{]}{[}0{]}

\_\_\_\_\_\_\_\_\_\_\_\_\_\_\_\_\_\_\_\_\_\_\_\_\_\_\_\_\_\_\_\_\_\_\_\_\_\_\_\_\_\_\_\_\_\_\_\_\_\_\_\_\_\_\_\_\_\_\_\_\_\_\_\_\_\_\_\_\_\_\_\_\_\_\_\_\_\_\_\_\_\_\_\_\_\_\_\_\_\_\_\_\_\_\_\_\_\_

concatenate\_2 (Concatenate) (None, 128, 128, 64) 0
batch\_normalization\_6{[}0{]}{[}0{]}

conv2d\_transpose\_2{[}0{]}{[}0{]}

\_\_\_\_\_\_\_\_\_\_\_\_\_\_\_\_\_\_\_\_\_\_\_\_\_\_\_\_\_\_\_\_\_\_\_\_\_\_\_\_\_\_\_\_\_\_\_\_\_\_\_\_\_\_\_\_\_\_\_\_\_\_\_\_\_\_\_\_\_\_\_\_\_\_\_\_\_\_\_\_\_\_\_\_\_\_\_\_\_\_\_\_\_\_\_\_\_\_

conv2d\_13 (Conv2D) (None, 128, 128, 32) 18464
concatenate\_2{[}0{]}{[}0{]}

\_\_\_\_\_\_\_\_\_\_\_\_\_\_\_\_\_\_\_\_\_\_\_\_\_\_\_\_\_\_\_\_\_\_\_\_\_\_\_\_\_\_\_\_\_\_\_\_\_\_\_\_\_\_\_\_\_\_\_\_\_\_\_\_\_\_\_\_\_\_\_\_\_\_\_\_\_\_\_\_\_\_\_\_\_\_\_\_\_\_\_\_\_\_\_\_\_\_

conv2d\_14 (Conv2D) (None, 128, 128, 32) 9248 conv2d\_13{[}0{]}{[}0{]}

\_\_\_\_\_\_\_\_\_\_\_\_\_\_\_\_\_\_\_\_\_\_\_\_\_\_\_\_\_\_\_\_\_\_\_\_\_\_\_\_\_\_\_\_\_\_\_\_\_\_\_\_\_\_\_\_\_\_\_\_\_\_\_\_\_\_\_\_\_\_\_\_\_\_\_\_\_\_\_\_\_\_\_\_\_\_\_\_\_\_\_\_\_\_\_\_\_\_

conv2d\_transpose\_3 (Conv2DTrans (None, 256, 256, 16) 2064
conv2d\_14{[}0{]}{[}0{]}

\_\_\_\_\_\_\_\_\_\_\_\_\_\_\_\_\_\_\_\_\_\_\_\_\_\_\_\_\_\_\_\_\_\_\_\_\_\_\_\_\_\_\_\_\_\_\_\_\_\_\_\_\_\_\_\_\_\_\_\_\_\_\_\_\_\_\_\_\_\_\_\_\_\_\_\_\_\_\_\_\_\_\_\_\_\_\_\_\_\_\_\_\_\_\_\_\_\_

concatenate\_3 (Concatenate) (None, 256, 256, 32) 0
batch\_normalization\_4{[}0{]}{[}0{]}

conv2d\_transpose\_3{[}0{]}{[}0{]}

\_\_\_\_\_\_\_\_\_\_\_\_\_\_\_\_\_\_\_\_\_\_\_\_\_\_\_\_\_\_\_\_\_\_\_\_\_\_\_\_\_\_\_\_\_\_\_\_\_\_\_\_\_\_\_\_\_\_\_\_\_\_\_\_\_\_\_\_\_\_\_\_\_\_\_\_\_\_\_\_\_\_\_\_\_\_\_\_\_\_\_\_\_\_\_\_\_\_

conv2d\_15 (Conv2D) (None, 256, 256, 16) 4624
concatenate\_3{[}0{]}{[}0{]}

\_\_\_\_\_\_\_\_\_\_\_\_\_\_\_\_\_\_\_\_\_\_\_\_\_\_\_\_\_\_\_\_\_\_\_\_\_\_\_\_\_\_\_\_\_\_\_\_\_\_\_\_\_\_\_\_\_\_\_\_\_\_\_\_\_\_\_\_\_\_\_\_\_\_\_\_\_\_\_\_\_\_\_\_\_\_\_\_\_\_\_\_\_\_\_\_\_\_

conv2d\_16 (Conv2D) (None, 256, 256, 16) 2320 conv2d\_15{[}0{]}{[}0{]}

\_\_\_\_\_\_\_\_\_\_\_\_\_\_\_\_\_\_\_\_\_\_\_\_\_\_\_\_\_\_\_\_\_\_\_\_\_\_\_\_\_\_\_\_\_\_\_\_\_\_\_\_\_\_\_\_\_\_\_\_\_\_\_\_\_\_\_\_\_\_\_\_\_\_\_\_\_\_\_\_\_\_\_\_\_\_\_\_\_\_\_\_\_\_\_\_\_\_

conv2d\_transpose\_4 (Conv2DTrans (None, 512, 512, 8) 520
conv2d\_16{[}0{]}{[}0{]}

\_\_\_\_\_\_\_\_\_\_\_\_\_\_\_\_\_\_\_\_\_\_\_\_\_\_\_\_\_\_\_\_\_\_\_\_\_\_\_\_\_\_\_\_\_\_\_\_\_\_\_\_\_\_\_\_\_\_\_\_\_\_\_\_\_\_\_\_\_\_\_\_\_\_\_\_\_\_\_\_\_\_\_\_\_\_\_\_\_\_\_\_\_\_\_\_\_\_

concatenate\_4 (Concatenate) (None, 512, 512, 16) 0
batch\_normalization\_2{[}0{]}{[}0{]}

conv2d\_transpose\_4{[}0{]}{[}0{]}

\_\_\_\_\_\_\_\_\_\_\_\_\_\_\_\_\_\_\_\_\_\_\_\_\_\_\_\_\_\_\_\_\_\_\_\_\_\_\_\_\_\_\_\_\_\_\_\_\_\_\_\_\_\_\_\_\_\_\_\_\_\_\_\_\_\_\_\_\_\_\_\_\_\_\_\_\_\_\_\_\_\_\_\_\_\_\_\_\_\_\_\_\_\_\_\_\_\_

conv2d\_17 (Conv2D) (None, 512, 512, 8) 1160
concatenate\_4{[}0{]}{[}0{]}

\_\_\_\_\_\_\_\_\_\_\_\_\_\_\_\_\_\_\_\_\_\_\_\_\_\_\_\_\_\_\_\_\_\_\_\_\_\_\_\_\_\_\_\_\_\_\_\_\_\_\_\_\_\_\_\_\_\_\_\_\_\_\_\_\_\_\_\_\_\_\_\_\_\_\_\_\_\_\_\_\_\_\_\_\_\_\_\_\_\_\_\_\_\_\_\_\_\_

conv2d\_18 (Conv2D) (None, 512, 512, 8) 584 conv2d\_17{[}0{]}{[}0{]}

\_\_\_\_\_\_\_\_\_\_\_\_\_\_\_\_\_\_\_\_\_\_\_\_\_\_\_\_\_\_\_\_\_\_\_\_\_\_\_\_\_\_\_\_\_\_\_\_\_\_\_\_\_\_\_\_\_\_\_\_\_\_\_\_\_\_\_\_\_\_\_\_\_\_\_\_\_\_\_\_\_\_\_\_\_\_\_\_\_\_\_\_\_\_\_\_\_\_

conv2d\_19 (Conv2D) (None, 512, 512, 1) 9 conv2d\_18{[}0{]}{[}0{]}

==================================================================================================\strut
\end{minipage}\tabularnewline
\bottomrule
\end{longtable}

\begin{enumerate}
\setcounter{enumi}{2}
\item
  参数调优

  由于模型较为复杂,常用的GridSearch与RandomSearch需要耗费大量时间,所有我们打算手动调参。

  主要参数:轮数初始为10,次数初始为32,个数初始为2。
\end{enumerate}

\begin{enumerate}
\def\labelenumi{\arabic{enumi}.}
\item
  、选择轮数(对全部数据训练的轮数)

  \includegraphics[width=2.85556in,height=2.14167in]{media/image17.png}\includegraphics[width=3.01389in,height=2.26042in]{media/image18.png}

  轮数:由loss图像所得,设置为5。
\item
  、选择次数(一轮中训练的次数)

  次数:次数=样本数/个数。
\item
  、选择个数(一次训练的样本数)

  个数:由于16张图片内存过大,而2张图片效果较差,所以设置为4。
\end{enumerate}
