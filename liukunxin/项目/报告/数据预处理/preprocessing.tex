% !Mode:: "TeX:UTF-8"

\chapter{数据预处理}

\section{数据格式}

根据我们得到的卫星图像船舶识别数据集,文件主要分为四个部分:

\begin{enumerate}
\def\labelenumi{\arabic{enumi}.}
\tightlist
\item
  sample\_submission\_v2.csv
\item
  test\_v2
\item
  train\_ship\_segmentations\_v2.csv
\item
  train\_v2
\end{enumerate}

分别对应样本提交集、测试集、验证集、训练集。

训练集约有200000张768x768的卫星船舶图像,近30G大小。

查看图像张数代码:

\begin{verbatim}
masks = pd.read_csv(os.path.join('../input/airbus-ship-detection/',
                                'train_ship_segmentations_v2.csv'))
print(masks.shape[0], 'masks found')
print(masks['ImageId'].value_counts().shape[0])
masks.head()
\end{verbatim}

结果如下

\begin{verbatim}
231723 masks found
192556
    ImageId         EncodedPixels
0   00003e153.jpg   NaN
1   0001124c7.jpg   NaN
2   000155de5.jpg   264661 17 265429 33 266197 33 266965 33 267733...
3   000194a2d.jpg   360486 1 361252 4 362019 5 362785 8 363552 10 ...
4   000194a2d.jpg   51834 9 52602 9 53370 9 54138 9 54906 9 55674 ...
\end{verbatim}

\section{RLE编码}

从上面代码我们可以看到train\_ship\_segmentations\_v2.csv是一个表格,表格有两列,分别是图像id和编码信息,这个编码方式是为RLE,英文名为run-length
encoding,翻译为游程编码,又译行程长度编码,又称变动长度编码法(run
coding),在控制论中对于二值图像而言是一种编码方法,对连续的黑、白像素数(游程)以不同的码字进行编码。游程编码是一种简单的非破坏性资料压缩法,其好处是加压缩和解压缩都非常快。其方法是计算连续出现的资料长度压缩之。

具体来说,RLE编码对图像每个像素点进行编号,用一个二元组(id,length)表示第id个像素点及往后走length长度的像素点都被标记了。这样编码可以大大节约内存,可以准确表示船舶被标注的位置。

\subsection{RLE解码}

关于RLE编码,将其解码成图片的代码如下:

\begin{verbatim}
# ref: https://www.kaggle.com/paulorzp/run-length-encode-and-decode
def rle_encode(img):
    '''
    img: numpy array, 1 - mask, 0 - background
    Returns run length as string formated
    '''
    pixels = img.T.flatten()
    pixels = np.concatenate([[0], pixels, [0]])
    runs = np.where(pixels[1:] != pixels[:-1])[0] + 1
    runs[1::2] -= runs[::2]
    return ' '.join(str(x) for x in runs)

def rle_decode(mask_rle, shape=(768, 768)):
    '''
    mask_rle: run-length as string formated (start length)
    shape: (height,width) of array to return 
    Returns numpy array, 1 - mask, 0 - background
    '''
    s = mask_rle.split()
    starts, lengths = [np.asarray(x, dtype=int) for x in (s[0:][::2], s[1:][::2])]
    starts -= 1
    ends = starts + lengths
    img = np.zeros(shape[0]*shape[1], dtype=np.uint8)
    for lo, hi in zip(starts, ends):
        img[lo:hi] = 1
    return img.reshape(shape).T  # Needed to align to RLE direction

def masks_as_image(in_mask_list):
    # Take the individual ship masks and create a single mask array for all ships
    all_masks = np.zeros((768, 768), dtype = np.int16)
    #if isinstance(in_mask_list, list):
    for mask in in_mask_list:
        if isinstance(mask, str):
            all_masks += rle_decode(mask)
    return np.expand_dims(all_masks, -1)
\end{verbatim}

以上代码包含了RLE格式的编码、解码,以及将RLE格式转换成一张图像。

//to-do

图展示了一张RLE解码后的图像,高亮部分表示被标记的船舶。

\section{剔除脏数据}

脏数据(Dirty
Read)是指源系统中的数据不在给定的范围内或对于实际业务毫无意义,或是数据格式非法,以及在源系统中存在不规范的编码和含糊的业务逻辑。

在数据处理中,对于一些对训练无用或者用处极小的数据进行筛除,从而提升训练效果。

本文发现,在文件大小小于50Kb时,较为频繁地出现无用的图像,所以在进行数据预处理的时候将文件大小小于50Kb的图像删掉了。

//to-do

图展示了部分非法的图像。

\section{分隔训练集和验证集}

以下代码实现了以下操作:

\begin{enumerate}
\def\labelenumi{\arabic{enumi}.}
\item
  统计出每张卫星图片囊括的船数
\item
  将有船和无船的图片分开,剔除图片文件小于50kb的图片
\item
  查看文件大小的分布以及处理后的数据

  masks{[}`ships'{]} = masks{[}`EncodedPixels'{]}.map(lambda c\_row: 1
  if isinstance(c\_row, str) else 0) unique\_img\_ids =
  masks.groupby(`ImageId').agg(\{`ships': `sum'\}).reset\_index()
  unique\_img\_ids{[}`has\_ship'{]} =
  unique\_img\_ids{[}`ships'{]}.map(lambda x: 1.0 if x\textgreater{}0
  else 0.0) unique\_img\_ids{[}`has\_ship\_vec'{]} =
  unique\_img\_ids{[}`has\_ship'{]}.map(lambda x: {[}x{]}) \# some files
  are too small/corrupt unique\_img\_ids{[}`file\_size\_kb'{]} =
  unique\_img\_ids{[}`ImageId'{]}.map(lambda c\_img\_id:
  os.stat(os.path.join(train\_image\_dir, c\_img\_id)).st\_size/1024)
  unique\_img\_ids =
  unique\_img\_ids{[}unique\_img\_ids{[}`file\_size\_kb'{]}\textgreater{}50{]}
  \# keep only 50kb files unique\_img\_ids{[}`file\_size\_kb'{]}.hist()
  masks.drop({[}`ships'{]}, axis=1, inplace=True)
  unique\_img\_ids.sample(5)
\end{enumerate}

结果如下:

\begin{verbatim}
        ImageId         ships   has_ship    has_ship_vec    file_size_kb
43824   3a704c694.jpg   0       0.0         [0.0]           134.526367
16053   155a58719.jpg   0       0.0         [0.0]           131.833984
95804   7f633690c.jpg   0       0.0         [0.0]           156.706055
129050  aba2ef0f5.jpg   9       1.0         [1.0]           208.094727
119822  9f5ef085c.jpg   0       0.0         [0.0]           186.836914
\end{verbatim}

//to-do

图展示了处理后文件大小的分布图,可以看出文件大小较为均匀地分布在120Kb左右。

以下代码用于以船只数为分层,7:3为比例,将原图像集分隔为训练集、验证集:

\begin{verbatim}
from sklearn.model_selection import train_test_split
train_ids, valid_ids = train_test_split(unique_img_ids, 
                test_size = 0.3, 
                stratify = unique_img_ids['ships'])
train_df = pd.merge(masks, train_ids)
valid_df = pd.merge(masks, valid_ids)
print(train_df.shape[0], 'training masks')
print(valid_df.shape[0], 'validation masks')
\end{verbatim}

运行结果如下:

\begin{verbatim}
161048 training masks
69034 validation masks
\end{verbatim}

得到161048训练用图像,69034验证用图像。

\subsection{基于船只数的重采样}

以下代码用于显示船只分布的直方图:

\begin{verbatim}
train_df['ships'].hist()
\end{verbatim}

结果如图,可以看出无船的图像数量明显多于有船的图像的数量,有少量船的图像数量明显多于有多数船的图像数量,这就导致一个问题------数据不平衡。

//to-do

数据不平衡对模型训练造成的影响。//to-do

本文将对训练数据进行重采样,代码如下:

\begin{verbatim}
train_df['grouped_ship_count'] = train_df['ships'].map(lambda x: (x+1)//2).clip(0, 7)
def sample_ships(in_df, base_rep_val=1500):
    if in_df['ships'].values[0]==0:
        return in_df.sample(base_rep_val//3) # even more strongly undersample no ships
    else:
        return in_df.sample(base_rep_val, replace=(in_df.shape[0]<base_rep_val))
    
balanced_train_df = train_df.groupby('grouped_ship_count').apply(sample_ships)
balanced_train_df['ships'].hist(bins=np.arange(10))
\end{verbatim}

采样后船只分布情况如图:

//to-do

可以看出,船只数量得到了较为均匀的分布,将更有助于训练结果。

\section{数据生成器}

由于卫星图像数据量过于庞大,一次性将所有训练图片读入内存进行训练,不仅内存吃紧,而且没有必要。正确做法应该是将数据分批次读入,每次读入指定数量张,分批训练,如此效果良好,而且解决了内存吃紧的问题。

如此操作需要编写数据生成器,在python和C\#语言中,对应的是yield关键字,用于产生一个generator生成器,代码如下:

\begin{verbatim}
def make_image_gen(in_df, batch_size = BATCH_SIZE):
    all_batches = list(in_df.groupby('ImageId'))
    out_rgb = []
    out_mask = []
    while True:
        np.random.shuffle(all_batches)
        for c_img_id, c_masks in all_batches:
            rgb_path = os.path.join(train_image_dir, c_img_id)
            c_img = imread(rgb_path)
            c_mask = masks_as_image(c_masks['EncodedPixels'].values)
            if IMG_SCALING is not None:
                c_img = c_img[::IMG_SCALING[0], ::IMG_SCALING[1]]
                c_mask = c_mask[::IMG_SCALING[0], ::IMG_SCALING[1]]
            out_rgb += [c_img]
            out_mask += [c_mask]
            if len(out_rgb)>=batch_size:
                yield np.stack(out_rgb, 0)/255.0, np.stack(out_mask, 0)
                out_rgb, out_mask=[], []

train_gen = make_image_gen(balanced_train_df)
train_x, train_y = next(train_gen)
print('x', train_x.shape, train_x.min(), train_x.max())
print('y', train_y.shape, train_y.min(), train_y.max())
\end{verbatim}

运行结果如下:

\begin{verbatim}
x (4, 768, 768, 3) 0.0 1.0
y (4, 768, 768, 1) 0 1
\end{verbatim}

可以看出每次取出4张图片,并且对图片的点值进行了归一化处理

以下代码用于验证数据生成器产生的结果:

\begin{verbatim}
fig, (ax1, ax2, ax3) = plt.subplots(1, 3, figsize = (30, 10))
batch_rgb = montage_rgb(train_x)
batch_seg = montage(train_y[:, :, :, 0])
ax1.imshow(batch_rgb)
ax1.set_title('Images')
ax2.imshow(batch_seg)
ax2.set_title('Segmentations')
ax3.imshow(mark_boundaries(batch_rgb, 
                        batch_seg.astype(int)))
ax3.set_title('Outlined Ships')
fig.savefig('overview.png')
\end{verbatim}

结果如图

//to-do

其中图一表示了卫星图像原图像,图二表示了在卫星图像的船只标注结果,图三表示了卫星图像原图像与标注结果相结合的图像。

以下代码用于生成验证集

\begin{verbatim}
valid_x, valid_y = next(make_image_gen(valid_df, VALID_IMG_COUNT))
print(valid_x.shape, valid_y.shape)
\end{verbatim}

运行结果如下:

\begin{verbatim}
(400, 768, 768, 3) (400, 768, 768, 1)
\end{verbatim}

\section{数据集增强}

数据集增强主要是为了减少网络的过拟合现象,通过对训练图片进行变换可以得到泛化能力更强的网络,更好的适应应用场景。

常用的数据集增强方法有:

\begin{itemize}
\tightlist
\item
  旋转\textbar{}反射变换(Rotation/reflection): 随机旋转图像一定角度;
  改变图像内容的朝向;
\item
  翻转变换(flip): 沿着水平或者垂直方向翻转图像;
\item
  缩放变换(zoom): 按照一定的比例放大或者缩小图像;
\item
  平移变换(shift): 在图像平面上对图像以一定方式进行平移;
\item
  可以采用随机或人为定义的方式指定平移范围和平移步长,
  沿水平或竖直方向进行平移. 改变图像内容的位置;
\item
  尺度变换(scale): 对图像按照指定的尺度因子, 进行放大或缩小;
  或者参照SIFT特征提取思想, 利用指定的尺度因子对图像滤波构造尺度空间.
  改变图像内容的大小或模糊程度;
\item
  对比度变换(contrast):
  在图像的HSV颜色空间,改变饱和度S和V亮度分量,保持色调H不变.
  对每个像素的S和V分量进行指数运算(指数因子在0.25到4之间), 增加光照变化;
\item
  噪声扰动(noise): 对图像的每个像素RGB进行随机扰动,
  常用的噪声模式是椒盐噪声和高斯噪声;
\item
  颜色变化:在图像通道上添加随机扰动。
\item
  输入图像随机选择一块区域涂黑,参考《Random Erasing Data
  Augmentation》。
\end{itemize}

以下代码用于对卫星图像进行数据集增强并生成对应的生成器:

\begin{verbatim}
from keras.preprocessing.image import ImageDataGenerator
dg_args = dict(featurewise_center = False, 
                samplewise_center = False,
                rotation_range = 15, # 旋转范围
                width_shift_range = 0.1, # 水平平移范围
                height_shift_range = 0.1, # 垂直平移范围
                shear_range = 0.01, # 透视变化的范围
                zoom_range = [0.9, 1.25], # 缩放范围
                horizontal_flip = True, # 水平翻转
                vertical_flip = True, # 垂直翻转
                fill_mode = 'reflect', # 填充模式
                data_format = 'channels_last') # 数据格式为通道last
# brightness can be problematic since it seems to change the labels differently from the images 
if AUGMENT_BRIGHTNESS:
    dg_args[' brightness_range'] = [0.5, 1.5] # 亮度范围
image_gen = ImageDataGenerator(**dg_args)

if AUGMENT_BRIGHTNESS:
    dg_args.pop('brightness_range')
label_gen = ImageDataGenerator(**dg_args)

def create_aug_gen(in_gen, seed = None):
    np.random.seed(seed if seed is not None else np.random.choice(range(9999)))
    for in_x, in_y in in_gen:
        seed = np.random.choice(range(9999))
        # keep the seeds syncronized otherwise the augmentation to the images is different from the masks
        g_x = image_gen.flow(255*in_x, 
                            batch_size = in_x.shape[0], 
                            seed = seed, 
                            shuffle=True)
        g_y = label_gen.flow(in_y, 
                            batch_size = in_x.shape[0], 
                            seed = seed, 
                            shuffle=True)

        yield next(g_x)/255.0, next(g_y)
\end{verbatim}

以下代码用于展示增强后的数据:

\begin{verbatim}
cur_gen = create_aug_gen(train_gen)
t_x, t_y = next(cur_gen)
print('x', t_x.shape, t_x.dtype, t_x.min(), t_x.max())
print('y', t_y.shape, t_y.dtype, t_y.min(), t_y.max())
# only keep first 9 samples to examine in detail
t_x = t_x[:9]
t_y = t_y[:9]
fig, (ax1, ax2) = plt.subplots(1, 2, figsize = (20, 10))
ax1.imshow(montage_rgb(t_x), cmap='gray')
ax1.set_title('images')
ax2.imshow(montage(t_y[:, :, :, 0]), cmap='gray_r')
ax2.set_title('ships')
\end{verbatim}

运行结果如下:

\begin{verbatim}
x (4, 768, 768, 3) float32 0.0 1.0
y (4, 768, 768, 1) float32 0.0 1.0
\end{verbatim}

//to-do

如图,可以看出图像被处理过的痕迹。
